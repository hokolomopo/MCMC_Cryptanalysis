\documentclass[11pt]{report}

\usepackage[utf8]{inputenc}
\usepackage[T1]{fontenc}
\usepackage[francais]{babel}

\usepackage{hyperref}
\usepackage{tikz}
\usepackage{amsmath,amssymb}
\usepackage{graphicx}
\usepackage{array}
\usepackage{color}
\usepackage{subfigure}
\usepackage{caption}
\usepackage{booktabs}
\usepackage{threeparttable}
\usepackage{amsmath}
\usepackage{array}
\usepackage{tabularx}
\usepackage{lmodern} % police Latin Modern
\usepackage{hyperref}
\usepackage{fancyhdr}
\usepackage{listingsutf8}
\usepackage{verbatim}
\usepackage[autolanguage]{numprint}
\usepackage{rotating}
\usepackage[top=3cm, bottom=3cm, left=3cm, right=3cm]{geometry}
\pagestyle{fancy}
\addto\captionsfrench{\def\tablename{Tableau}}
\lhead{Stegen Thomas s154315 \quad  Adrien Minne s154340 \quad  Delaunoy Arnaud s153059}
\hypersetup{                    % parametrage des hyperliens
    colorlinks=true,                % colorise les liens
    breaklinks=true,                % permet les retours à la ligne pour les liens trop longs
    urlcolor= black,                 % couleur des hyperliens
    linkcolor= black,                % couleur des liens internes aux documents (index, figures, tableaux, equations,...)
    citecolor= green                % couleur des liens vers les references bibliographiques
    }
\definecolor{mygreen}{RGB}{28,172,0} % color values Red, Green, Blue
\definecolor{mylilas}{RGB}{170,55,241}
\title{Processus stochastiques: cryptanalyse}
\author{Stegen Thomas s154315 \\ Adrien Minne s154340 \\ Delaunoy Arnaud s153059}
\date{}
\renewcommand\thesection{\arabic{section}}
\renewcommand\thesubsection{\arabic{section}.\arabic{subsection}}
\renewcommand\thesubsubsection{Question \arabic{subsubsection}}
\begin{document}
\setcounter{secnumdepth}{3}
\lstset{
language=matlab,
keywordstyle=\color{blue},
morekeywords=[2]{1}, keywordstyle=[2]{\color{black}},
identifierstyle=\color{black},
stringstyle=\color{mylilas},
commentstyle=\color{mygreen},
numberstyle={\tiny \color{black}},
emph=[1]{for,end,break},emphstyle=[1]\color{blue},
}
 \lstset{%
            inputencoding=utf8,
                breaklines=true,
                extendedchars=true,
                literate=%
                {é}{{\'e}}{1}%
                {è}{{\`e}}{1}%
                {à}{{\`a}}{1}%
                {ç}{{\c{c}}}{1}%
                {œ}{{\oe}}{1}%
                {ù}{{\`u}}{1}%
                {É}{{\'E}}{1}%
                {È}{{\`E}}{1}%
                {À}{{\`A}}{1}%
                {Ç}{{\c{C}}}{1}%
                {Œ}{{\OE}}{1}%
                {Ê}{{\^E}}{1}%
                {ê}{{\^e}}{1}%
                {î}{{\^i}}{1}%
                {ô}{{\^o}}{1}%
                {û}{{\^u}}{1}%
                {ë}{{\¨{e}}}1
                {û}{{\^{u}}}1
                {â}{{\^{a}}}1
                {Â}{{\^{A}}}1
                {Î}{{\^{I}}}1
        }

\begin{titlepage}
\maketitle
\end{titlepage}
\section{ Première partie : chaines de Markov pour la modélisation du langage et MCMC}
\subsection{Chaine de Markov pour la modélisation du langage}
\subsubsection{}
L'élément (i,j) de la matrice de transition correspond à la probabilité de passer de l'état i à l'état j. Il correspond donc à la probabilité que la lettre i soit suivie de la lettre j dans la séquence. Dès lors, soit $\theta$ l'élément (i,j) de la matrice de transition, $\theta$ est le paramètre d'une loi de Bernouilli avec comme possibilités:
\begin{itemize}
\item l'élément i est suivi de j (avec une probabilité $\theta$)
\item l'élément i n'est pas suivi de j (avec une probabilité $1-\theta$)
\end{itemize}
\paragraph{}
La méthode du maximum de vraisemblance consiste à maximiser $P(\bf{D}_n|\theta)$ avec $\bf{D}_n$ l'échantillon de donnée, ici seq1 et n le nombre de données. Pour une variable de Bernouilli, on a: Soit,\\ 
$$
\mbox{} x_i = \left\{
    \begin{array}{ll}
        1 & \mbox{si i est suivi de j} \\
        0 & \mbox{si i n'est pas suivi de j}
    \end{array}
\right.
$$
et m le nombre d’occurrences de la lettre i.
\begin{align*}
P(\bf{D}_n|\theta) &= \prod_{i = 1}^{m} (x_i \theta +(1-x_i)(1-\theta)) \\
                   &= \theta^{n_1} (1-\theta)^{n_0}
\end{align*}
Avec $n_0$ le nombre de fois où $x_i = 0$ et $n_1$ le nombre de fois où $x_i = 1$. Déterminons maintenant le $\theta$ maximisant cette fonction:
\begin{align*}
\frac{\partial P(\bf{D}_n|\theta)}{\partial \theta} 
&= n_1 \theta^{n_1-1} (1-\theta)^{n_0} - n_1\theta^{n_1} (1-\theta)^{n_0-1} \\
&= \theta^{n_1-1} (1-\theta)^{n_0-1} (n_1 (1-\theta) - n_0 \theta) \\
&= \theta^{n_1-1} (1-\theta)^{n_0-1} (n_1 - \theta (n_1 + n_0))
\end{align*}
La valeur de $\theta$ maximisant la fonction $P(\bf{D}_n|\theta)$ est donc:
$$ \theta_{i,j} = \frac{n_1}{n_0+n_1} = \frac{\mbox{nombre d'occurrences de i suivies de j}}{\mbox{nombre d'occurrences de i}} $$
\subsubsection{}
\subsubsection{}
\subsubsection{}
\subsubsection{}
\subsection{Algorithme MCMC}
\subsubsection{}
Pour prouver que $\pi_0$ est une distribution stationnaire de la chaine de Markov, il suffit de prouver que $\pi_0=\pi_0 *Q$. 
On sait que les équations de balances détaillées $\pi_0(i)Q_{i,j} = \pi_0(j)Q_{j,i}$ sont satisfaites.
Passant en notation indicielle, on doit donc montrer que :
\begin{align*}
\pi_0(i) &= \sum^{N}_{k=0} \pi_0(k)*Q_{k,i} \\
& = \sum^{N}_{k=0} \pi_0(i)*Q_{i,k} \\
& = \pi_0(i) \sum^{N}_{k=0} \pi_0(i) Q_{i,k} \\
& = \pi_0(i)
\end{align*}
Cette distribution stationnaire est unique si la matrice de transition Q est irréductible.
\subsubsection{}
\underline{\bf{mettre la référence du pdf.}}\\
Étudions d'abord la probabilité de transition.\\
La probabilité d'obtenir un élément $x_j$ sachant que l'élément précédent de la chaîne de Markov est $x_i$ est pour $i \neq j$ la probabilité que cet élément soit généré selon la loi q et accepté.
$$P(x_j | x_i) = \alpha(x_j, x_i) q(x_j| x_i)$$
avec
\begin{align*}
\alpha(x_j, x_i) 
&= min \left\{1, \frac{f(x_j)}{f(x_i)} \frac{q(x_i|x_j)}{q(x_j|x_i)}\right\}\\
&= min \left\{1, \frac{cP_X(x_j)}{cP_X(x_i)} \frac{q(x_i|x_j)}{q(x_j|x_i)}\right\}
\end{align*}
La probabilité d'obtenir à nouveau l'élément $x_i$ sachant que l'élément précédent de la chaîne de Markov est également $x_i$ est la somme de la probabilité que l'élément $x_i$ soit généré selon la loi q et accepté et de la probabilité que tout autre élément soit généré et refusé.
$$P(x_i | x_i) = \alpha(x_i, x_i) q(x_i|x_i) + \sum_k (1-\alpha(x_k, x_i)) q(x_k |x_i)$$

Dans le cas où l'élément généré est différent du précédent, on a:
\begin{align*}
P(x_j | x_i) \pi_0(x_i)
&= \alpha(x_j, x_i) q(x_j| x_i) \pi_0(x_i) \\
&= min \left\{1, \frac{cP_X(x_j)}{cP_X(x_i)} \frac{q(x_i|x_j)}{q(x_j|x_i)}\right\} q(x_j| x_i) \pi_0(x_i) \\
&= \frac{\pi_0(x_i)}{cP_X(x_i)} min \left\{cP_X(x_i) q(x_j|x_i), cP_X(x_j) q(x_i|x_j)\right\} \\
&\text{en posant } x_i \leftarrow x_j \text{ et } x_j \leftarrow x_i \\
&= \frac{\pi_0(x_j)}{cP_X(x_j)} min \left\{cP_X(x_j) q(x_i|x_j), cP_X(x_i) q(x_j|x_i)\right\} \\
&= min \left\{1, \frac{cP_X(x_i)}{cP_X(x_j)} \frac{q(x_j|x_i)}{q(x_i|x_j)}\right\} q(x_i| x_j) \pi_0(x_j)\\
&= \alpha(x_i, x_j) q(x_i| x_j) \pi_0(x_j) \\
&= P(x_i | x_j) \pi_0(x_j)
\end{align*} 

Dans le cas où l'élément généré est le même que le précédent, $x_i$ étant égal à $x_j$ il est évident que 
$$P(x_i | x_j) \pi_0(x_j) = P(x_j | x_i) \pi_0(x_i)$$
car 
$$P(x_i | x_i) \pi_0(x_i) = P(x_i | x_i) \pi_0(x_i)$$
\section{Deuxième partie : décryptage d’une séquence codée}
\subsubsection{}
\subsubsection{}
\subsubsection{}
\subsubsection{}
\subsubsection{}
\end{document}